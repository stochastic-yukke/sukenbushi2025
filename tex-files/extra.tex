\section{参考文献}\label{bib}
\cite{yukie}-\cite{tsuji}は,この記事を書く上で(多少なりとも)参考にした資料。一方,\cite{suron}-\cite{alg}は,記事を書く上でほとんど参照していないが,読者の参考になるかもしれない文献。

\cite{modern_integers}は,初等整数論に始まり,代数的整数論や解析的整数論の基礎的な内容と展望が,テンポよく述べられている本である。理想数のアイデアが詳しく載っていた。
\cite{susemi}は,文章が千夜一夜物語を意識した小説風になっている,楽しい記事である。\cite{intro}は,「類体論の心と七五三の心は通ずるものがあると思ったのです」など印象的な部分が多く,わかりやすく伝えるにはどうすれば良いか考える際に参考になった。
\cite{suron}は,どこ/いつだったかは忘れたが,一部読んだことがあった本。「岩波講座 現代数学の基礎」シリーズの『数論1』『数論2』をまとめたもの。絶版になっているので,どこかの図書館で借りるか神田の明倫館書店で買うかしよう。\cite{primes_p}は,AKS判定法についての論文。

今回の記事執筆を通して,図書館というインフラの重要性を認識した。

\begin{thebibliography}{9}
    \bibitem{yukie} 雪江明彦,『整数論1 初等整数論から$p$進数へ』,日本評論社(2013).
    \bibitem{modern_integers} 落合理,『現代整数論の風景 素数からゼータ関数まで』,日本評論社(2019).
    \bibitem{susemi} 原隆,『数学セミナー』「数の世界の千一夜」第3回,第5回-第7回,日本評論社(2022).
    \bibitem{intro} 加藤和也,『数論への招待』,丸善出版(2012).
    \bibitem{tsudoi} YouTube『すうがく徒のつどい@オンライン「代数的整数論 類体論入門」』,alg-d,\\
    \url{https://www.youtube.com/watch?v=MtFluwn36bk}
    \bibitem{primes_p} Manindra Agrawal, Neeraj Kayal, Nitin Saxena, \emph{PRIMES is in P}, Annals of Mathematics, \textbf{160} (2004), 781-793,\\
    \url{https://annals.math.princeton.edu/wp-content/uploads/annals-v160-n2-p12.pdf}
    \bibitem{frobenius_kurims} 越川皓永,『Frobenius写像の周辺』第42回 京都大学数理解析研究所 数学入門講座(2021),\\
    \url{https://www.kurims.kyoto-u.ac.jp/~kenkyubu/kokai-koza/R3-koshikawa.pdf}
    \bibitem{tsuji} 『素イデアル分解法則を考える(ヒルベルトの理論とフロベニウス自己同型)』,tsujimotterのノートブック,\\
    \url{https://tsujimotter.hatenablog.com/entry/hilbert-theorem}
    \bibitem{suron} 加藤和也, 黒川信重, 斎藤毅,『数論I Fermatの夢と類体論』,岩波書店(2005).
    \bibitem{yellow} 著:J.ノイキルヒ,監訳:足立恒雄,訳:梅垣敦紀,『代数的整数論』,丸善出版(2012).
    \bibitem{yukie2} 雪江明彦,『整数論2 代数的整数論の基礎』,日本評論社(2013).
    \bibitem{alg} alg-d,『代数的整数論 類体論』(2013),\\
    \url{https://alg-d.com/math/number\_theory/class\_field\_theory.pdf}
\end{thebibliography}