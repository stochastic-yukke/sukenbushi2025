\section{いくつかの儚い夢たち}
\begin{dfn}[数の集合]
    正の整数全体の集合を$\mathbb{Z^{+}}$と書く.
\end{dfn}

ここでは,偽であるような\textsf{FD}を幾つか挙げます。\textsf{FD}と仲よくなるとともに,\textsf{FD}が成り立つのは結構特殊な場合であることをわかって欲しいです。読者も色々な例を試してみて下さい。これ以降,すべて常体で書いていきます(別に高圧的な態度を取ろうとしているわけじゃないよ)。

\begin{mybox}[展開公式]
    $x,y \in \mathbb{R},\, p \in \mathbb{Z^{+}}$に対して,$(x + y)^p = x^p + y^p$は,自明な場合を除いて決して成立しない。
\end{mybox}

\begin{mybox}[平方根]
    $p = 1/2, \, x,y \in \mathbb{C}$のとき,定義\ref{freshman}の主張は,$\sqrt{x+y} = \sqrt{x} + \sqrt{y}$ということになり,一般に成り立たないことは明白である。また,$x,y$が非負実数であるとき$\sqrt{x+y},\sqrt{x}+\sqrt{y}\geqq 0$だから,$x+y\leqq x+2\sqrt{x}\sqrt{y}+y$より,任意の非負実数$x,y$に対して$\sqrt{x+y}\leqq\sqrt{x}+\sqrt{y}$である。
\end{mybox}

\begin{mybox}[集合の直積]
    $x,y$を集合とする。集合$\mathrm{A}$に対して,$n \in \mathbb{Z^{+}}$に対する$\mathrm{A}^n$は,$\mathrm{A}$の$n$個の直積を表すものとし,集合の和の記号$\cup$ を+と書くことにすると,$(a_1,\ldots,a_p)\in(x+y)^p\Leftrightarrow\forall i\in\{1,\ldots,p\},(a_i\in x\text{または} a_i\in y)\overset{\not\Rightarrow}{\Leftarrow}(\forall i\in\{1,\ldots,p\}, a_i\in x)\text{または}(\forall i\in\{1,\ldots,p\}, a_i\in y)\Leftrightarrow (a_1,\ldots,a_p)\in x^p+y^p$より,$(x+y)^p\supset x^p+y^p$となる。したがって,\textsf{FD}は一般に成り立たない。
\end{mybox}

\begin{mybox}[排他的論理和]
    $1\oplus1=0,\,0\oplus0=0,\,1\oplus0=1,\,0\oplus1=1$として$\oplus$を定義する。整数$n,m$に対し,二進法表記した$i$桁目$n_i,m_i$について$n_i\oplus m_i$を計算し,その結果を$i$桁目とした数を$n\oplus m$とかく。例えば$3\oplus6=0011_{(2)}\oplus0110_{(2)}=0101_{(2)}=5$となる。$(3\oplus6)^2\ne 3^2\oplus6^2$など,\textsf{FD}は一般に成り立たない。
\end{mybox}

一応,\textsf{FD}が成り立つ例も1つだけ示しておく。

\begin{mybox}[min-plus代数(トロピカル幾何学)では\dots]
    $\mathbb{R}$に対して,$x\oplus y=\min\{x, y\}$,$x\odot y=x+y$という演算を導入する。また,$p\in\mathbb{Z^+}$に対して,$x^p=x\odot x\odot\cdots\odot x=px$する。すると,任意の$x,y\in\mathbb{R},\, p\in\mathbb{Z}^+$に対して,$(x\oplus y)^p=\min\{x,y\}\times p=\min\{px, py\}=x^p\oplus y^p$となる。
\end{mybox}

他にも,\textsf{FD}が成り立たない場合,成り立つ場合,について色々計算してみて欲しい。


