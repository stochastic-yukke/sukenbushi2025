\section{参考にした文献と参考になるであろう文献}\label{bib}
\cite{yukie}--\cite{tsuji}は,この記事を書く上で(多少なりとも)参考にした資料。一方,\cite{suron}--\cite{frak3}は,記事を書く上でほとんど参照していないが,読者の参考になるかもしれない文献。

\cite{modern_integers}は,初等整数論に始まり,代数的整数論や解析的整数論の基礎的な内容と展望が,テンポよく述べられている本である。理想数のアイデアが詳しく載っていた。
\cite{susemi}は,文章が千夜一夜物語を意識した小説風になっている,楽しい記事である。\cite{intro}は,「類体論の心と七五三の心は通ずるものがあると思ったのです」など印象的な部分が多く,わかりやすく伝えるにはどうすれば良いか考える際に参考になった。
\cite{primes_p}は,AKS判定法についての論文。\cite{suron}は,どこ/いつだったかは忘れたが,一部読んだことがあった本。「岩波講座 現代数学の基礎」シリーズの『数論1』『数論2』をまとめたもの。絶版になっているので,どこかの図書館で借りるか神田の明倫館書店で買うかしよう。
\cite{frak1}は,ドイツの書字をめぐって,歴史や教育について述べている。\cite{frak2}は,印刷博物館の「黒の芸術」展に寄せて学芸員の方が書いたもの。urlにwwwを含めないと開けないようだ。\cite{frak3}には,SütterlinによるKurrent(ドイツの筆記体)やFrakturをはじめとする古典的なドイツの書字が使われなくなったことなどについて,歴史と考察が書かれている。

今回の記事執筆を通して,図書館というインフラの重要性を認識した。

すべて,最終閲覧日は2026年1月3日である。また,「参考になるであろう文献」すなわち\cite{suron}以降の文献は,変更したり追加したりする可能性がある。

\begin{thebibliography}{9}
%%% 今回はbibtexを使わなかった %%%
    \bibitem{yukie} 雪江明彦,『整数論1 初等整数論から$p$進数へ』,日本評論社(2013).
    \bibitem{modern_integers} 落合理,『現代整数論の風景 素数からゼータ関数まで』,日本評論社(2019).
    \bibitem{susemi} 原隆,『数学セミナー』「数の世界の千一夜」第3回,第5回--第7回,日本評論社(2022).
    \bibitem{intro} 加藤和也,『数論への招待』,丸善出版(2012).
    \bibitem{tsudoi} YouTube『すうがく徒のつどい@オンライン「代数的整数論 類体論入門」』,alg-d,\\
    \url{https://www.youtube.com/watch?v=MtFluwn36bk}
    \bibitem{primes_p} Manindra Agrawal, Neeraj Kayal, Nitin Saxena, \emph{PRIMES is in P}, Annals of Mathematics, \textbf{160} (2004), 781--793,\\
    \url{https://annals.math.princeton.edu/wp-content/uploads/annals-v160-n2-p12.pdf}
    \bibitem{frobenius_kurims} 越川皓永,『Frobenius写像の周辺』第42回 京都大学数理解析研究所 数学入門講座(2021),\\
    \url{https://www.kurims.kyoto-u.ac.jp/~kenkyubu/kokai-koza/R3-koshikawa.pdf}
    \bibitem{tsuji} 『素イデアル分解法則を考える(ヒルベルトの理論とフロベニウス自己同型)』,tsujimotterのノートブック,\\
    \url{https://tsujimotter.hatenablog.com/entry/hilbert-theorem}
    \bibitem{suron} 加藤和也, 黒川信重, 斎藤毅,『数論I Fermatの夢と類体論』,岩波書店(2005).
    \bibitem{yellow} 著:J.ノイキルヒ,監訳:足立恒雄,訳:梅垣敦紀,『代数的整数論』,丸善出版(2012).
    \bibitem{yukie2} 雪江明彦,『整数論2 代数的整数論の基礎』,日本評論社(2013).
    \bibitem{alg} alg-d,『代数的整数論 類体論』(2013),\\
    \url{https://alg-d.com/math/number\_theory/class\_field\_theory.pdf}
    \bibitem{frak1} 深井里奈子,『ドイツにおける筆記体の変遷と手書き文字に求められる役割 : 初等学校の書字教育を参考に』,千葉大学大学院人文社会科学研究科研究プロジェクト報告書,\textbf{299} (2016), 86--99,\\
    \url{https://opac.ll.chiba-u.jp/da/curator/100347/BA31027730_299_p086_FUK.pdf}
    \bibitem{frak2} 式洋子,印刷博物館ニュース Vol.96 特集1『特集「黒の芸術」解題』,印刷博物館(2025),\\
    \url{https://www.printing-museum.org/etc/pnews/09601.php}
    \bibitem{frak3} ``A History of the Old German Script'', Walden Font Co.(2019),\\
    \url{https://blog.waldenfont.com/2019/06/26/the-history-of-the-old-german-script/}
\end{thebibliography}

\section{追記}
p.10のフラクトゥーアについて書いた部分の正しさを支える参考文献がないことに気づきました。おそらく,自分の記憶だけをもとに書いてしまったと思われます。書いてあることは概ね正しかったのですが,参考文献を掲載した方が良いと感じたので,追加しました。言葉遣いも断定調でなく「とされている」というふうにしておきました。

フラクトゥーアに関する参考文献は,\cite{frak1}--\cite{frak3}です。